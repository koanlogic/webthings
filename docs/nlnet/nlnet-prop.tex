%!TEX TS-program = xelatex
%!TEX encoding = UTF-8 Unicode

\documentclass[12pt]{article}
\usepackage{geometry}
\geometry{a4paper}
\usepackage[parfill]{parskip}
\usepackage[pdftex]{graphicx}
\usepackage{amssymb}
\usepackage[english]{babel}
\usepackage[colorlinks=true,linkcolor=blue]{hyperref}
\usepackage[official]{eurosym}

%\usepackage{fontspec,xltxtra,xunicode}
%\defaultfontfeatures{Mapping=tex-text}
%\setromanfont[Mapping=tex-text]{Hoefler Text}
%\setsansfont[Scale=MatchLowercase,Mapping=tex-text]{Gill Sans}
%\setmonofont[Scale=MatchLowercase]{Andale Mono}
\hyphenpenalty=1000

\newcommand{\swmod}[1]{\mbox{\texttt{#1}}}

\title{KINK @ NLNET}
\author{Pierpaolo Giacomin\footnote{Independent consultant.}, Mirko Rossini\footnote{CS Dept. University of Bologna.},\\Thomas Fossati and Steven Dorigotti\footnote{KoanLogic SRL.}}

\begin{document}
\maketitle
\tableofcontents
\newpage

\paragraph{Abstract}
\emph{KINK is a project which aims at bridging the ``big'' Internet with the Internet of Things by producing open standards -- mainly within the IETF CoRE Working Group -- and corresponding \mbox{opensource} implementations.}\\

\section{Introduction}
\label{sec:intro}

Figure \ref{fig:arch} illustrates the system's high-level architecture. KINK is a logical module which may reside on a standalone device or can be integrated into Customer Premise Equipment. Its function is to allow communication between HTTP, the most common Application Layer protocol of the Internet, and CoAP, a similar RESTful transfer protocol which has been designed by the IETF to be applied to particularly constrained scenarios such as typical Wireless Sensors Networks. Such generic design can find an infinite number of applications, ranging from domotics to medical, automotive and agricultural industries, only to mention a few. The diagram shows some more specific examples, such as an Energy provider gathering information from Smart Meters in a home network via the Internet $[2]$, or intelligent nodes communicating independently in a Machine 2 Machine configuration $[4]$.

\begin{center}
    \begin{figure}
        \includegraphics[width=14cm]{../share/images/kink-homenet}
            \caption{KINK architecture}
            \label{fig:arch}
    \end{figure}
\end{center}

The current KINK partners are: the Computer Science Department at the University of Bologna and KoanLogic SRL, an Italian company devoted to \mbox{opensource} and open standards implementation, and owner of a number of public GPL and BSD licensed tools\footnote{See \href{http://koanlogic.com}{http://koanlogic.com} and \href{https://github.com/koanlogic}{https://github.com/koanlogic} for details.}.

We are seeking financial aid to help us reach KINK's first deadline: the implementation of a proxy module to enable transparent communication between the unconstrained and constrained sides of the future Internet.

\section{Related Work}
We started by looking into a number of readily available opensource options when initially considering the assembly of the HTTP-CoAP proxy.  In this section, each of these components is briefly illustrated, and the main reasons for deciding against their adoption is discussed.

After careful analysis, the choice was made to build upon the \href{http://libevent.org}{libevent} core, since its non-blocking, event-driven architectural model was deemed ideal for the proxy architecture.  Furthermore, it provides an HTTP interface via the evhttp module, and has a built-in non-blocking DNS resolver, which could be easily extended to supply DNS-SD capabilities -- these having the greatest importance in mapping the embedded resource discovery functions of CoAP to the unconstrained \mbox{Internet}.

\subsection{Apache Web Server}
The Apache project is mature, robust and is well maintained and supported. Apache 2.0 provides a valid and tested HTTP implementation, fitting one of the project needs.  Being a web server, Apache is already optimized to handle a huge number of HTTP requests.  Furthermore, Apache is easily extensible through a simple module system.  However, it has a couple of fundamental drawbacks:
\begin{enumerate}
\item being a mature project, thought for the web, Apache has an enormous list of features that are not needed for the KINK case that are likely to cause an excessive memory footprint for the intended KINK target platform (i.e. a typical OpenWRT equipped CPE box);
\item being an HTTP Origin Server, its architectural model doesn't fit the Intermediary use case, where the ``one thread per request'' policy may lead to devastating performance drops, especially with long-standing requests required by the CoAP Observe mapping.
\end{enumerate}

\subsection{Varnish}
Varnish is an HTTP accelerator with 5+ years on the trenches of web sites such as Facebook and Globo.  Varnish can be extended with new modules. Although this feature lacks extensive documentation, example modules can be found on the project website, and can be used as a starting point for module developers.
Varnish could be a good starting point for the HTTP/CoAP proxy, as it provides a tested HTTP protocol implementation and an optimized caching system, however, like Apache, it assumes modern, performant hardware, which could be one or more orders of magnitude bigger than the KINK typical target platform.

\subsection{Squid 3}
Squid 3 is a proxy server and web cache daemon.  It is rich of features and well tested under incredibly heavy loads (e.g. it provides the frontend for Wikipedia's entire public infrastructure).
Unlike Varnish, Squid has no plugin interface, and needs some quite intrusive tweaks to handle the CoAP bits.  Namely, one has to heavily patch the \texttt{FwdState} and \texttt{HttpStateData} classes for handling the protocol flow and needed translations; furthermore, the cache system has to be extended to implement CoAP's caching and freshness maintenance policies.
Additionally, the same overall ``defect'' of Apache and Varnish was found also in Squid: the general purpose nature of the software architecture leads to inflation in the memory footprint, making it unfit for the embedded nature of the KINK target.

\subsection{LibCoAP}
The \href{http://sourceforge.net/projects/libcoap}{libcoap} project is a CoAP implementation in C written by Olaf Bergmann at TZI.
It was initially considered as the building block for the CoAP protocol handler, but it proved difficult to integrate into the libevent model.  Another motivation to drop it in favour of evcoap is that we have a strong requirement on controlling the CoAP bits as the specification is still evolving, and further, extensions to CoAP are likely to be implemented in order to experiment with corner cases in the HTTP-CoAP mapping function.

\subsection{Others}
There are no opensource projects with the same scope as KINK that we know of, and this was the primary motivation to start working on it.
Currently, the market offers few proprietary solutions, e.g. Sensinode's \href{http://www.sensinode.com/EN/products/nanoservice.html}{NanoService}.  Huge players like Huawei, Ericsson and InterDigital are working on their own commercial platforms but the state-of-art is not publicly available.


\section{Work Items}
The bulk of current activity is centered around mapping HTTP and CoAP, the two main application protocols available on each communication segment.

As described in Section \ref{sec:intro}, the overall goal is to provide native communication between humans and things through seamless integration of physical objects into the Web platform.

The work items depicted in the following subsections are currently under active development, or are planned to start soon.

\subsection{CoAP Implementation}
The \href{https://github.com/koanlogic/webthings/tree/master/bits/evcoap}{\swmod{evcoap}} module fully implements the CoAP protocol as per \href{http://tools.ietf.org/html/draft-ietf-core-coap}{draft-ietf-core-coap-08} with server, client and proxy roles.

The deadline for this module is stringent as our participation at \href{http://www.etsi.org/plugtests/coap/coap.htm}{ETSI CoAP plugtests} is scheduled for the end of March 2012.

The module provides a C library based on the reactor pattern which builds on Niels Provos' \href{http://libevent.org}{libevent} and KoanLogic \href{http://koanlogic.com/libu}{libu}, and adds components (e.g. an embeddable resource-based file system) to ease the creation of sophisticated CoAP agents.

\subsection{HTTP-CoAP Mapping I-D}
The I-D \href{http://tools.ietf.org/html/draft-castellani-core-http-mapping}{draft-castellani-core-http-mapping} is a joint effort of \mbox{KoanLogic}, \mbox{Ericsson}, \mbox{InterDigital} and the Engineering Department at University of Padua (under the IETF CoRE Working Group umbrella), to provide the architectural and implementation guidance for a KINK-like component.

As such it represents a fundamental deliverable of the project as a whole, and at the same time it receives vital feedback from the implementation experience gained while developing the KINK software modules.

\subsection{Agnostic Caching}
The \href{https://github.com/koanlogic/webthings/bits/kache}{\swmod{kache}} module implements a specialized cache aimed at storing informational resources, be it sensor generated data stream or a typical HTTP resource representation, in an agnostic way. (TODO Mirko).


\subsection{CoAP Proxy Extensions for Sleepy Sensors I-D}
The \href{https://github.com/koanlogic/webthings/blob/master/docs/draft-core-monitor-option/}{draft-fossati-core-monitor-option} I-D introduces a couple of new CoAP Options that can be harnessed by ``sleepy'' sensors to force the Proxy to act as a blackboard/bin object where they can publish/retrieve resources needed for their operation, during their typically very short duty cycles.  This is a convenient feature which we are going to present at \mbox{IETF 83}, and implement in the KINK Proxy.



\subsection{HTTP-CoAP Proxy}
The only activity that has not yet begun, is the implementation of the proxy module, which constitutes the first milestone of the KINK project. 

The related high level architectural design has been concluded as part of Mirko Rossini's MS thesis work.  The evcoap, kache, libu and libevent's evhttp modules provide fundamental building blocks that will be reused to match KINK's application logic.

\subsection{System}
\href{https://github.com/koanlogic/webthings/tree/master/sys}{\swmod{sys}} is an early stage component which will include firmware generation procedures and documentation targeted at both the involved platforms:
\begin{itemize}
\item sys/kink: an \href{http://openwrt.org}{OpenWrt}-based firmware generation system for the target hardware (currently Ubiquiti Routerstation Pro) and possible kernel patches;
\item sys/wsn: a set of scripts and HowTos which aid in setting up a constrained network based on the target sensor nodes (currently Zolertia Z1 platform). \href{http://www.tinyos.net}{TinyOS} and \href{http://www.contiki-os.org}{Contiki} are the two OS-level candidates for the base system. At this stage both of them are being followed since CoAP support is still at an early stage on both of them. Custom extensions/patches may be required. TinyOS and Contiki also include simulation tools which will be explored for massive testing purposes.
\end{itemize}


\section{Dissemination}
KoanLogic has gained a considerable amount experience from promoting its own open source projects in the past, and will apply similar methodologies to KINK. The main dissemination media are:
\begin{itemize}
\item open source conferences. As an example, KLone has been presented at OSCON and CONFSL. If accepted, we plan to present KINK at FOSDEM in 2013 (an attempt was made this year, but the project wasn't mature enough to achieve a demo of the system);
\item IETF. In the past years, first with LibSCS then through participation in the CoRE Working Group, KoanLogic has come into close contact with the IETF, will participate at the CoAP plug test in March 2012, and possibly prepare presentations on KINK;
\item dissemination via contacts and presentations at University of Bologna and Padua;
\item the main communication channel for OSS projects (i.e. the Web), first of all by providing easy access to all resources (source code repositories, Wikis, tutorials, mailing lists, IRC, etc), then by putting several SEO optimisation techniques into practice.
\end{itemize}

Furthermore, KINK will be released under a liberal license, most likely BSD, in order to maximise the interested audience: not only hobbyists, but also major players in the embedded industry.

\section{Expected Effort}
The effort needed to complete the KINK development tasks is summarized in the following table -- the unit of measure is man/month referred to a senior resource.

\begin{center}
\begin{tabular}{|l|c|c|c|c|c|}
	\hline 
	  & Design & Development & Module Test & Integration Test & Total \\
	\hline
	spec (IETF) & 1     & 1     & -     & -     & 2 \\
	\hline 
	evcoap      & -     & 1.5   & 0.5   & -     & 2 \\
	\hline
	kache       & -     & 1     & 0.5   & -     & 1.5 \\
	\hline
	kink        & 1     & 7.5   & -     & 4     & 12.5 \\
	\hline
	sys         & -     & ?     & ?     & ?     & ? \footnotemark \\
	\hline
	\multicolumn{5}{|c|}{} & 18 \\
	\hline
\end{tabular}
\end{center}
\footnotetext{sys requires ongoing effort which is difficult to estimate due to both hardware and OS-level constraints - its cost will be absorbed by KoanLogic.}

The total remaining expected cost is $18 \times 5000$\euro~$=~90000$\euro~ (excluding the sys component).

To date, the KINK project has been financed in toto by KoanLogic, which plans to further fund half of the remaining effort (i.e. 45000\euro).

If the proposal is accepted, 15000\euro~are expected to come through the EU-financed BOOSTER project.

We are asking NLNET for a 30000\euro~funding to complete the budget.

\section{Follow-ups}

The completion of the previously mentioned development tasks constitutes the core logic of the KINK project, and the software framework will already be usable and customisable by third parties for their deployments. Some envisioned requirements for the next-step hardware product based on KINK are:

\begin{itemize}
\item to specialise the sys OpenWRT-based distribution for hardware with dedicated 802.15.4 support;
\item integration of a routing protocol such as RPL;
\item resource discovery via DNS-SD or other;
\item system configuration web application.
\end{itemize}

All of the above aiming at a user-friendly customisable open-black-box system.

\end{document}
